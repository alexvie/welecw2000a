
\documentclass[11pt,ngerman,american,a4paper,dvipdfm]{paper}
% article style
%   - 11pt Schriftgroesse
%   - new austrian (neue Rechtschreibung)
%   - Papierformat A4
%   - pdf-hyperlinks
%   - separate Titelseite (ohne Seitennummer) kann durch 'titlepage' erreicht werden

% Packages
% --------
\usepackage[ansinew]{inputenc}
\usepackage[T1]{fontenc}
\usepackage{a4}

%verschiedene Schriftarten
%-------------------------
\usepackage{palatino}

\usepackage{babel}
\usepackage[pdftex]{graphicx,color} % for PDF
%\usepackage[dvips]{graphicx,color} % for DVI (standard)
%\usepackage{graphics}
%\input{rgb}						% some colors
\usepackage{framed}     
\usepackage{fancyhdr}   
\usepackage{listings}   
\usepackage{verbatim}   
\usepackage{enumerate}  
\usepackage{caption}		

% Seitenspiegel
% -------------
\setlength{\textwidth}{17cm}				
\setlength{\textheight}{24cm}				
\setlength{\topmargin}{-15mm}				
\setlength{\oddsidemargin}{-5mm}		
\setlength{\evensidemargin}{-5mm}	
\pagestyle{fancy}

% Kopfzeile
% ---------
\lhead{{\footnotesize{Alexander Lindert}}} \chead{{\footnotesize{W2000L Oscilloscope Reference Manual}}}

\rhead{{\footnotesize{\today}}}
\setlength{\headheight}{24.5pt}
\setlength{\headsep}{5mm}
% Fusszeile
% ---------
\lfoot{}
\cfoot{\footnotesize{Seite \thepage}}
\rfoot{}        
%\setlength{\footrulewidth}{0.4pt}            % Linie �ber der Fusszeile

% ---------------
\usepackage{setspace}   
\singlespace           
%\onehalfspace         
%\doublespace          

% Beginn des Dokumentes
% ---------------------
\begin{document}

\selectlanguage{ngerman}

% for plain text files 
%\newcommand{\txtinput}[1]{
%		\footnotesize
%		\verbatiminput{#1}
%		\normalsize		
%		}

  \lstset{
    basicstyle=\ttfamily\scriptsize,
    numbers=left,
    frame=single,
    tabsize=4,
    extendedchars=true,
    %backgroundcolor=\color{grey},
		keywordstyle=\color{blue},
		%identifierstyle=, 								% nothing happens
		commentstyle=\color{CommentGreen}, 
		stringstyle=\color{red}, 				
		showstringspaces=false, 					% no special string spaces
    language=matlab
  } 
  \definecolor{CommentGreen}{rgb}{0,0.78,0.2}

%\includegraphics[width=0.70\textwidth]{expl-mc.jpg}
%\lstinputlisting[caption=VHDL-Code]{../src/BMC.vhd}
%\lstinputlisting[caption=PSL-Assertions in V-Unit]{../src/BMC.psl}
%\lstinputlisting[caption=Schaltung in SMV]{../src/BMC.smv}
%\lstinputlisting[language=Clean,caption=NuSMV Ausgabe (Auszug)]{smv-output.txt}
%\lstinputlisting [breaklines=true, caption=\lstname,captionpos=b]{../ue02.m}
%\includegraphics[width=0.6\textwidth]{../strecke.pdf}

\section {Preface}
This document does describe the open source oscilloscope specific periphials implemented in the FPGA of the Welec W2000A series oscilloscopes.
As CPU the FPGA design does use the open source LEON3 SparcV8 processor from \href{www.gaisler.com}.
All parts from \href{www.gaisler.com} are not documented here. If you need information of these parts read the grip.pdf and the Sparcv8.pdf reference manuals.
\subsection{GPL}
 
\section{Clock domain solution}

\end{document}
